\documentclass [a3paper,portrait,20pt]{article}
\usepackage[utf8]{inputenc}
\usepackage{titlesec}
\usepackage[english]{babel}
\usepackage{fontspec}
\usepackage{xcolor}
\usepackage{fancyhdr}
\usepackage [margin=0.6in]{geometry}
\usepackage{graphicx}
\usepackage{hyperref}
\usepackage{lipsum}
\usepackage{multicol}
\usepackage{anyfontsize}

\titlespacing\section{0pt}{12pt plus 4pt minus 2pt}{0pt plus 8pt minus 6pt}
\setmainfont[Color=003f7f]{AnglicanText}

\begin{document}

\setlength{\columnsep}{8mm}
\setlength{\parskip}{0.1em}
\pagestyle{empty}
\title{\fontsize{100}{120}\selectfont{The Oath of Covenant of the Azure Concord}}
\author{\Huge Ashe of Criamon}
\date{\Huge 21\textsuperscript{st} December 1231}
\maketitle
\fontspec[Color=003366]{Precious}
\Large{{Being the Charter of the Covenant of The Azure Concord in the 1231\textsuperscript{st} Year of Our Lord, Jesus Christ.}}

\vspace*{5mm}
\noindent
\Large{I pledge my lifelong support and loyalty to the Covenant of the Azure Concord, and declare that the
	trials and fortunes of this covenant are now my own. Just as I am pledged to the Oath of Hermes, so
	do I pledge the covenant to the Order of Hermes, and the authority of the Tribunal of Transylvania.
	I swear to uphold and protect this covenant regardless of personal price. Over all the years of my
	life, and throughout my studies and travels, I will neither betray the covenant nor give aid to its
	enemies. In times of need, I will aid the covenant in whatever way I am able, and I will devote 
	myself to its service if the need is clear. I will abide by the decisions of the ruling council of
	this covenant, and I will treat these decisions as if they were my own. I will treat my fellows with
	respect and fairness, and I will not attempt to harm them in any way. Their blood is my blood. Where
	the covenant stands, there do I stand; how the covenant grows, so do I grow; should the covenant
	fall, then do I fall. This I so swear, upon the honour of my house and its Founder.}

\huge{Given under our hand,}

\vspace{105mm}


\fontspec[Color=000000]{BlackChancery}
\begin{multicols}{2}
\section*{\fontsize{35}{40}\selectfont{Membership}}
\thispagestyle{empty}
\begin{small}
	The covenant allows for two types of membership of its council, and recognizes a third status, which it
	offers to visitors to the covenant.
	
	The status of Protected Guest may be extended to any person by the formal invitation of a single full 
	member of the covenant. Protected Guests are afforded the basic rights detailed by this charter, and 
	are not obligated to the Council of Members, nor are they a member of this council. Protected Guests 
	may partake in meetings of the Council of Members should they desire it, but are required to leave if 
	asked to do so by a member of that council, and are afforded no voice nor vote unless granted such by 
	the council’s chairman, the disceptator. The status of Protected Guest may be may be revoked by the 
	member who granted it, or by a vote of the Council of Members.
	
	The status of Probationary Member of the Council may be extended to any magus in good standing of the 
	Order of Hermes, who owes no allegiance nor fealty to any other covenant, and is admitted upon the 
	unanimous approval of the current Council of Members. Provisional members assume the basic and 
	provisional rights detailed by this charter and the duties therein attached. The status of provisional 
	member shall last a period of seven seasons, unless abridged through censure or cancelled through 
	expulsion.
	
	The status of Full Member of the Council is extended upon the completion of the duties and obligations 
	of a probationary member, unless testimony is brought against him that proves him unfit to swear the 
	Oath of Covenant in good conscience; in which case all rights of membership will be withdrawn. Should 
	elevation to the role of full member take place, then all rights and duties of probationary membership 
	are shed, to be replaced with the assumption of the basic and full rights detailed by this charter, and 
	the duties therein attached. Full membership persists, unless abridged through censure or cancelled 
	through expulsion.
	
	Should a magus ever come to desire release from this covenant, he must renounce his Oath of Covenant in
	the presence of at least two members of the council, and shall thereby be relieved of all duties and 
	rights, and may not call upon such rights furthermore.
	
\end{small}
\end{multicols}
\newpage

\begin{multicols}{2}

\begin{small}
\section*{\fontsize{35}{40}\selectfont{Governance of this Covenant}}
	The members of this covenant are governed by the Council of Members, which shall consist of all 
	probationary and full members of the covenant. This council shall not declare action except on behalf 
	of the entire membership of the covenant; no action may be demanded of individuals by council 
	agreement. Conversely, the rulings of the council cannot be overturned by an individual.
	
	Any member of the covenant shall have the right and duty to convene the Council of Members for 
	consideration of matters justly grave, and all members shall be charged with attendance and diligence 
	in the proceedings. Should it not be possible to convene the full Council of Members, any quorum 
	consisting of more than half of its current members including the Councillor for the matter raised is 
	considered valid, or failing that, his chosen prot\'eg\'e; else the discharge of the council’s duty 
	must be delayed until such time as the full council may be convened, or the Councillor (or his 
	prot\'eg\'e) may attend. The Council of Members shall convene four times each year, one day prior to
	each equinox and solstice, regardless of call from any member, and all members of the covenant should
	endeavour to make themselves present.
	
	Motions to be decided upon by the Council of Members must be introduced by a member; debated fully and
	justly, allowing those who wish to speak to do so; and then proposed for the vote. Proposals must be
	seconded by another member of the covenant, else no vote can take place. All issues shall be passed by
	a majority vote of the members there present; excepting that the unanimous opinion of the Council of
	Members is required for issues involving changes to the charter; expulsion of a member; and acceptance
	of a new probationary member.
	
	Each Councillor will be allowed the right to veto any motion that falls within his bounds of duty, so
	long as the motion has been debated fully, and the Councillor feels that the motion would create a
	significant risk to the Covenant,  or his realm. He must, however, suggest an acceptable alternative.\footnote{Exemplar Gratis - A motion is brought to
		the council regarding the release of a captured magical creature. The Coucillor of Security may
		employ his Security veto if he thinks that the released creature might bring undue attention to
		the Covenant, but is required to suggest an alternative to releasing it freely.}
	
	The Council of Members shall confer the office of disceptator to the representative of the covenant in
	matters of governance and temporal concern. The title of disceptator is a duty of each and every full
	member of the covenant; this position is cyclical and mandatory, with the responsibility rotating in
	sequence of Hermetic seniority amongst the full members of the council. Each disceptator serves for a
	period of seven years, commencing one year following Tribunal. The duties of the disceptator are: to
	attend regular meetings of the council; to keep order at meetings; to break tied votes with a
	discretionary casting vote; to determine the yearly surplus of provision and store; and to act as a
	spokesman for the Council of Members. The disceptator shall not be empowered to rule on matters on the
	covenant's behalf, but instead is charged with ensuring the rulings of the Council of Members are
	enacted.
	
\section*{\fontsize{35}{40}\selectfont{Resources Owned by this Covenant}}
	Resources of this covenant are held in common by the Council of Members, and it is the responsibility
	of this council to maintain and defend them.
	
	This covenant lays claim to all the vis originating from undisputed and unclaimed sources discovered by
	members of the council; save for the first harvest of a new vis source, which belongs to the finder or
	finders. This covenant also lays claim to any vis gifted to the Council of Members as a whole. In all
	other situations, undisputed and unclaimed vis belongs to the finder or finders.
	
	This covenant lays claim to all books obtained by members of the council while acting at the behest of
	the council, and all books scribed by members of the council where payment was received for this
	scribing from the covenant's resources. This covenant also lays claim to any texts gifted to the Council
	of Members as a whole.
	
	This council lays claim to all magical items obtained by members of the council while acting at the
	behest of the council; and all magical items made by the members of the council where payment was
	received for this manufacture from the covenant's resources. This covenant also lays claim to any
	magical items gifted to the Council of Members as a whole.
	
	This council lays claim to all monies generated using the resources of the covenant. This council also
	lays claim to all monies obtained by members of the council while acting at the behest of the council.
	This covenant also lays claim to any monies gifted to the Council of Members as a whole.
	
	This council lays claim to all buildings, defenses, chattels, and inhabitants of the covenant. This
	council also lays claims to any such buildings, defenses, chatells and inhabitants gifted to the
	Council of Members as a whole.
	
	Surplus resources of the covenant will be determined at the Winter meeting of the Council of Members.
	Resources necessary for the continued existence of the covenant and the protection of its members'
	rights are accounted for first; this includes payment for seasons of work performed on behalf of the
	covenant, and a stipend of vis for the casting of the Aegis of the Hearth. Contributions to all debts
	owed to the covenant are decided by the disceptator, and set aside. The remaining resources are deemed
	surplus, and shall be allocated to the settlement of requests from each member of the covenant.
	
\section*{\fontsize{35}{40}\selectfont{Rights of the Members of this Covenant}}
	Each and every member of this covenant and protected guests shall be intitled to the basic rights of
	the covenant; to whit, full and unrestricted access to the protection and support of the covenant within
	the boundaries of the covenant by all the rights and benefits accorded by the Code of Hermes, the
	benefit of a sanctum which shall remain inviolate and the supply of materials thereof, access to the
	library of the covenant, and victuals appropriate to the status of a magus. These basic rights shall not
	be abridged except by expulsion from the Council of Members.
	
	In furtherance and additional to the basic rights, a full member of the covenant shall be entitled to
	the full rights of the covenant; to whit, the right to presence and a vote in the Council of Members,
	which he shall exercise dutifully with due prudence. Further, full and unrestricted access to the
	services and skills of the servants and covenfolk. Further, an equal right to all surplus provision and
	store necessary to conduct his studies, or the travel demanded by those studies; such rights to include
	(but be not limited to) vis, monies, and diverse magical and mundane resources claimed by the covenant.
	Where a conflict is evident between members of the council over the allotment of surplus resource,
	distribution is drawn by ballot; excepting that priority claims that have been advanced and agreed by
	the disceptator are taken into consideration prior to the ballot. These rights shall not be abridged
	except by decision of the council under conditions of grave concern.
	
	In furtherance and additional to the basic rights of a member of this covenant, a probationary member
	of the covenant is entitled the probationary rights of the covenant; to whit, a fractional share of
	those rights and duties offered to a full member of the covenant, such share being equal to half that
	offered to full members. A probationary member's vote counts only half that of a full member, and they
	may only claim half the share of the surplus provision and store of the covenant's resources afforded a
	full member. The services and skills of the servants and covenfolk may not be halved, but the needs of a
	full member of the covenant take precedence over the needs of a probationary member. Further, a
	probationary member of the covenant who remains true to his Oath of Covenant has the right to remain at
	the covenant for a total of seven seasons following the conferral of this status. Further, a
	probationary member of the covenant has the right to be considerd for full membership of this covenant
	after serving a total of seven seasons as a probationary member. These rights shall not be abridged
	except by decision of the council under conditions of grave concern.
	
\section*{\fontsize{35}{40}\selectfont{Responsibilities of the Members of this Covenant}}
	Members of this covenant are obligated to obey the Oath of Hermes and the Peripheral Code, as demanded
	by the Oath of Covenant; failure on this account will not be tolerated by the Counil of Members, and the
	covenant reserves the right to censure those members who are convicted in just Tribunal of an offence
	against the Order of Hermes.
	
	The responsibility of members of this covenant towards its lasting success is dependant on service to
	the covenant. The Council of Members will declare the duties that need be performed at the regular
	meetings of the covenant. Such duties include (but are not limited to) the safeguarding and harvesting
	of the covenant's claimed vis sources, the safeguarding and harvesting of the covenant's income, the
	wellbeing and discipline of the covenant's employees, the maintenance of the covenant's resources, the
	increase of the covenant's resources, and the maintenance of good relations with the covenant's allies.
	Duties that will not entail more than a week of service at low personal risk will be assigned by the
	council to its members, with no more than one being assigned to each member in each season. Such
	assigned duties attract no recompense or advantage to the member who discharges them, but cannot be
	refused without reasonable extenuation.
	
	Duties that will entail a higher investment of time or personal risk will be offered up for service by
	the covenant. These services will attract a renumeration which shall be commensurate with the time,
	risk, and potential benefit to the covenant. The renumeration is decided by the disceptator, but
	maintains a minimum payment which shall be, for a single season of work at low risk with a low gain, two
	pawns of vis, of the flavour most prelavent in the stores at the time. The disceptator may increase the
	renumeration to increase the attractiveness of a particular urgent task, for the Council of Members is
	not empowered to force a member to accept one of these duties unless failure to perform it would be in
	breach of this covenant, in which case the threat of censure may be employed. All payments will be made
	in the Spring meeting of the Council. If there is more than one claimant for the service, and each
	claimant refuses to share the duty, then the disceptator will assign the duty by ballot. If there is
	insufficient vis to meet the demands of the council, the disceptator may withhold payment for one or
	more years. Covenant work may be declared such retroactively.
	
	Each probationary member of the covenant is obligated to perform no fewer than one of the tasks under
	renumeration currently outlined by the Council of Members during his period of probation. For this
	mandatory service, no payment need be offered by the Council of Members.
\section*{\fontsize{35}{40}\selectfont{Censure of the Members of this Covenant}}
	If a member should contravene the decisions of the council, by vote or by charter, then the member may
	be censured or expelled by a vote of the Council of Members. Censure requires the passing of a motion at
	a meeting of the Council of the Members. The censure of a full member revokes the rights of that status,
	returning him to probationary status; whereupon he assumes all the duties and rights of that status.
	Censure must not prejudice the application of a probationary member to the position of full member of
	the covenant. The censure of a probationary member shall abridge the rights of the member to remain a
	probationary member of the covenant, and shall confer upon him instead the status of Protected Guest.
	The status of a Protected Guest may be withdrawn at any time by a vote of the Council of Members without
	need for censure.
	
	Expulsion is enacted by a unanimous vote of the remaining Council of Members. Expulsion is the only
	means through which a member of the covenant shall lose his basic rights; and requires that the former
	member ceases to draw upon those basic rights subsequent to the first full moon after expulsion was
	enacted. Should a magus be cast out of the Order, it is the duty and obligation of the covenant that he
	shall also and without delay be expelled from the covenant.
	\vfil 
		\vskip-\prevdepth \nointerlineskip\null 
	\penalty 200  \vfilneg
\end{small}
\end{multicols}
\section*{\centering\fontsize{40}{40}\selectfont{The Seal of Office}}
\vspace{37mm}
\begin{multicols}{3}
\section*{\fontsize{12}{12}\selectfont{This charter was approved by, }}
\columnbreak
\section*{\fontsize{12}{12}\selectfont{Quaesitor in good standing, in the}}
\columnbreak
\section*{\fontsize{12}{12}\selectfont{Year of Our Lord, Jesus Christ.}}
\end{multicols}
\end{document}